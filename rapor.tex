\documentclass[12pt,a4paper]{article}
\usepackage[utf8]{inputenc}
\usepackage[turkish]{babel}
\usepackage{graphicx}
\usepackage{geometry}
\usepackage{hyperref}
\usepackage{listings}
\usepackage{xcolor}
\usepackage{float}
\usepackage{booktabs}
\usepackage{array}
\usepackage{longtable}
\usepackage{tikz}
\usetikzlibrary{shapes.geometric, arrows, positioning}

\geometry{margin=2.5cm}

% Kod blokları için stil
\lstset{
    basicstyle=\ttfamily\small,
    backgroundcolor=\color{gray!10},
    frame=single,
    breaklines=true,
    captionpos=b
}

\begin{document}

%-------------------------------------------
% KAPAK SAYFASI
%-------------------------------------------
\begin{titlepage}
\centering
\vspace*{2cm}

	{\Huge\textbf{Borç Takip Projesi}\par}
	\vspace{0.5cm}
	{\LARGE Mobil Programlama Projesi Raporu\par}

	\vspace{3cm}

	{\Large Ahmet Serdar Turan\par}
	{\large Öğrenci No: 200101001\par}

	\vspace{2cm}

	{\large Danışman: Fatima Bhutta\par}

	\vspace{1cm}

	{\large Bölüm: Bilgisayar Mühendisliği\par}
	{\large İstanbul Rumeli Üniversitesi\par}

	\vfill

	{\large Ocak 2026\par}
\end{titlepage}

%-------------------------------------------
% İÇİNDEKİLER
%-------------------------------------------
\tableofcontents
\newpage

%-------------------------------------------
% 1. PROJENİN AMACI
%-------------------------------------------
\section{Projenin Amacı}

Günlük hayatta bireyler arasında gerçekleşen borç alışverişleri sıklıkla unutulmakta veya yanlış hatırlanmaktadır. Arkadaşlar, aile üyeleri veya iş arkadaşları arasındaki maddi ilişkilerin takibi, özellikle birden fazla kişiyle eş zamanlı borç ilişkisi olan kullanıcılar için karmaşık bir hal alabilmektedir.

\textbf{Borç Takip} uygulaması, bu problemi çözmek amacıyla geliştirilmiştir. Uygulama, kullanıcıların hem alacaklarını hem de borçlarını dijital ortamda kayıt altına almalarını, takip etmelerini ve yönetmelerini sağlamaktadır.

\subsection{Uygulamanın Sunduğu Faydalar}

\begin{itemize}
    \item \textbf{Merkezi Kayıt Sistemi:} Tüm borç ve alacakların tek bir platformda toplanması
    \item \textbf{Hatırlatma Mekanizması:} Ödeme tarihlerinin yaklaştığında bildirim gönderilmesi
    \item \textbf{Finansal Özet:} Anlık bakiye durumunun görüntülenmesi
    \item \textbf{Döviz Dönüştürme:} TL tutarların USD karşılığının otomatik hesaplanması
    \item \textbf{Veri Dışa Aktarımı:} PDF ve CSV formatlarında rapor oluşturma
    \item \textbf{Kolay Paylaşım:} Borç bilgilerinin QR kod veya mesaj ile paylaşılması
\end{itemize}

%-------------------------------------------
% 2. PROJENİN HEDEFİ
%-------------------------------------------
\section{Projenin Hedefi}

Bu projenin temel hedefi, kullanıcıların kişisel borç ve alacak ilişkilerini \textbf{kolay, hızlı ve güvenilir} bir şekilde takip edebilecekleri bir Android mobil uygulaması geliştirmektir.

\subsection{Alt Hedefler}

\begin{enumerate}
    \item Kullanıcı dostu bir arayüz ile borç ekleme, düzenleme ve silme işlemlerinin gerçekleştirilmesi
    \item SQLite veritabanı kullanılarak verilerin cihaz üzerinde güvenli bir şekilde saklanması
    \item AlarmManager ile zamanında hatırlatma bildirimlerinin gönderilmesi
    \item REST API entegrasyonu ile güncel döviz kuru bilgisinin alınması
    \item Material Design 3 prensiplerine uygun modern bir kullanıcı deneyimi sunulması
    \item Verilerin PDF ve CSV formatlarında dışa aktarılabilmesi
\end{enumerate}

%-------------------------------------------
% 3. UYGULAMANIN GENEL MİMARİSİ
%-------------------------------------------
\section{Uygulamanın Genel Mimarisi}

Uygulama, Android geliştirme standartlarına uygun olarak \textbf{Activity} tabanlı bir mimari üzerine inşa edilmiştir. Veri yönetimi için \textbf{Model-View-Controller (MVC)} benzeri bir yaklaşım benimsenmiştir.

\subsection{Kullanılan Activity Yapısı}

Uygulama toplam \textbf{4 Activity} içermektedir:

\begin{table}[H]
\centering
\caption{Uygulama Activity Yapısı}
\begin{tabular}{|l|l|p{6cm}|}
\hline
\textbf{Activity} & \textbf{Rol} & \textbf{Açıklama} \\
\hline
DashboardActivity & Launcher & Ana sayfa, özet kartları ve toplam bakiye görüntüleme \\
\hline
MainActivity & Liste Görünümü & Alacak/Borç listesi, arama, filtreleme, sıralama \\
\hline
AddDebtActivity & Ekleme Formu & Yeni borç/alacak kaydı oluşturma \\
\hline
DebtDetailActivity & Detay Görünümü & Kayıt detayları, ödendi işaretleme, silme, paylaşım \\
\hline
\end{tabular}
\end{table}

\subsection{Veri Akışı}

% ---- GÖRSEL 1: Mimari Diyagramı ----
% Buraya bir mimari akış diyagramı eklenecek
% Önerilen içerik: Activity'ler arası geçiş, DatabaseHelper ile iletişim,
% CurrencyService API çağrısı gösterimi
\begin{center}
\fbox{\parbox{0.8\textwidth}{\centering\vspace{2cm}\textbf{[GÖRSEL 1: Mimari Diyagram]}\\\small Activity'ler arası veri akışı ve bileşen ilişkileri\vspace{2cm}}}
\end{center}

\subsection{Önemli Sınıflar ve Modüller}

\begin{table}[H]
\centering
\caption{Temel Sınıflar ve Görevleri}
\begin{tabular}{|l|l|p{5.5cm}|}
\hline
\textbf{Paket} & \textbf{Sınıf} & \textbf{Görev} \\
\hline
database & Debt.java & Borç veri modeli (POJO) \\
\hline
database & DatabaseHelper.java & SQLite CRUD işlemleri \\
\hline
adapter & DebtAdapter.java & RecyclerView adaptörü \\
\hline
service & CurrencyService.java & Döviz kuru API servisi \\
\hline
notification & NotificationHelper.java & Alarm ve bildirim yönetimi \\
\hline
notification & AlarmReceiver.java & Alarm tetikleyici \\
\hline
notification & BootReceiver.java & Cihaz açılışında alarm yenileme \\
\hline
util & PreferenceManager.java & SharedPreferences yönetimi \\
\hline
\end{tabular}
\end{table}

\subsection{Bileşen Diyagramı}

\begin{center}
\begin{tikzpicture}[
    node distance=1.5cm,
    box/.style={rectangle, draw, rounded corners, minimum width=2.5cm, minimum height=1cm, align=center, fill=blue!10},
    dbbox/.style={rectangle, draw, rounded corners, minimum width=2.5cm, minimum height=1cm, align=center, fill=green!10},
    apibox/.style={rectangle, draw, rounded corners, minimum width=2.5cm, minimum height=1cm, align=center, fill=orange!10},
    arrow/.style={->, thick}
]

% Activities
\node[box] (dashboard) {Dashboard\\Activity};
\node[box, right=of dashboard] (main) {Main\\Activity};
\node[box, right=of main] (add) {AddDebt\\Activity};
\node[box, right=of add] (detail) {DebtDetail\\Activity};

% Database
\node[dbbox, below=2cm of main] (dbhelper) {Database\\Helper};
\node[dbbox, below=of dbhelper] (sqlite) {SQLite\\Database};

% Services
\node[apibox, below=2cm of add] (currency) {Currency\\Service};
\node[apibox, below=of currency] (api) {Exchange\\Rate API};

% Arrows
\draw[arrow] (dashboard) -- (main);
\draw[arrow] (main) -- (add);
\draw[arrow] (main) -- (detail);
\draw[arrow] (dashboard) -- (dbhelper);
\draw[arrow] (main) -- (dbhelper);
\draw[arrow] (add) -- (dbhelper);
\draw[arrow] (detail) -- (dbhelper);
\draw[arrow] (dbhelper) -- (sqlite);
\draw[arrow] (dashboard) -- (currency);
\draw[arrow] (detail) -- (currency);
\draw[arrow] (currency) -- (api);

\end{tikzpicture}
\end{center}

%-------------------------------------------
% 4. KULLANILAN ARAÇLAR VE TEKNOLOJİLER
%-------------------------------------------
\section{Kullanılan Araçlar ve Teknolojiler}

\subsection{Geliştirme Ortamı}

\begin{itemize}
    \item \textbf{Android Studio:} Hedgehog | 2023.1.1 sürümü
    \item \textbf{Gradle:} Kotlin DSL ile yapılandırma (build.gradle.kts)
    \item \textbf{Git \& GitHub:} Versiyon kontrolü ve kaynak kod yönetimi
\end{itemize}

\subsection{Programlama Dilleri ve SDK}

\begin{itemize}
    \item \textbf{Java 11:} Ana programlama dili
    \item \textbf{Android SDK:} Compile SDK 36, Min SDK 24, Target SDK 36
    \item \textbf{XML:} Layout ve kaynak dosyaları
\end{itemize}

\subsection{Kütüphaneler ve Bağımlılıklar}

\begin{table}[H]
\centering
\caption{Kullanılan Kütüphaneler}
\begin{tabular}{|l|l|p{5cm}|}
\hline
\textbf{Kütüphane} & \textbf{Sürüm} & \textbf{Kullanım Amacı} \\
\hline
AndroidX AppCompat & 1.7.0 & Geriye dönük uyumluluk \\
\hline
Material Components & 1.12.0 & Material Design 3 bileşenleri \\
\hline
Retrofit & 2.9.0 & REST API istemcisi \\
\hline
Gson & 2.10.1 & JSON dönüştürme \\
\hline
ZXing Android & 4.3.0 & QR kod oluşturma \\
\hline
ConstraintLayout & 2.2.1 & Esnek layout tasarımı \\
\hline
RecyclerView & 1.4.0 & Liste görünümü \\
\hline
CardView & 1.0.0 & Kart bileşenleri \\
\hline
\end{tabular}
\end{table}

\subsection{API ve Veri Kaynakları}

\begin{itemize}
    \item \textbf{ExchangeRate-API:} Güncel USD/TRY döviz kuru
    \item \textbf{Endpoint:} \texttt{https://api.exchangerate-api.com/v4/latest/USD}
\end{itemize}

%-------------------------------------------
% 5. ARAYÜZ (UI/UX) TASARIMI
%-------------------------------------------
\section{Arayüz (UI/UX) Tasarımı}

Uygulama arayüzü, \textbf{Material Design 3} tasarım ilkelerine uygun olarak geliştirilmiştir. Kullanıcı deneyimi odaklı, sade ve anlaşılır bir tasarım hedeflenmiştir.

\subsection{Renk Paleti}

\begin{table}[H]
\centering
\caption{Uygulama Renk Paleti}
\begin{tabular}{|l|l|l|}
\hline
\textbf{Kullanım} & \textbf{Renk} & \textbf{Hex Kodu} \\
\hline
Ana Renk (Primary) & Mavi & \#1976D2 \\
\hline
Alacak (Pozitif) & Yeşil & \#4CAF50 \\
\hline
Borç (Negatif) & Kırmızı & \#F44336 \\
\hline
Net Bakiye & Mavi & \#2196F3 \\
\hline
İkincil Metin & Gri & \#757575 \\
\hline
Arka Plan & Açık Gri & \#F5F5F5 \\
\hline
\end{tabular}
\end{table}

\subsection{Ekran Görüntüleri}

\subsubsection{Dashboard (Ana Sayfa)}

% ---- GÖRSEL 2: Dashboard Ekran Görüntüsü ----
\begin{center}
\fbox{\parbox{0.5\textwidth}{\centering\vspace{4cm}\textbf{[GÖRSEL 2: Dashboard Ekranı]}\\\small Net bakiye, alacak ve borç kartları\vspace{4cm}}}
\end{center}

Dashboard ekranı, kullanıcının finansal durumunu tek bakışta görmesini sağlar:
\begin{itemize}
    \item Net bakiye (TL ve USD karşılığı)
    \item Toplam alacaklar (yeşil kart)
    \item Toplam borçlar (kırmızı kart)
\end{itemize}

\subsubsection{Liste Görünümü (MainActivity)}

% ---- GÖRSEL 3: Liste Ekranı ----
\begin{center}
\fbox{\parbox{0.5\textwidth}{\centering\vspace{4cm}\textbf{[GÖRSEL 3: Liste Ekranı]}\\\small Alacaklar/Borçlar tabları ve liste görünümü\vspace{4cm}}}
\end{center}

Liste ekranı özellikleri:
\begin{itemize}
    \item TabLayout ile Alacaklar/Borçlar ayrımı
    \item Arama çubuğu
    \item Sıralama ve filtreleme butonları
    \item PDF/CSV dışa aktarım
    \item Floating Action Button ile hızlı ekleme
\end{itemize}

\subsubsection{Borç Ekleme Formu}

% ---- GÖRSEL 4: Ekleme Formu ----
\begin{center}
\fbox{\parbox{0.5\textwidth}{\centering\vspace{4cm}\textbf{[GÖRSEL 4: Ekleme Formu]}\\\small Kişi adı, tutar, tarih, hatırlatma seçenekleri\vspace{4cm}}}
\end{center}

Ekleme formu bileşenleri:
\begin{itemize}
    \item TextInputLayout ile kişi adı ve tutar girişi
    \item DatePicker ile tarih seçimi
    \item Vade tarihi seçeneği
    \item Tekrarlayan borç seçenekleri (Haftalık/Aylık)
    \item Takvime ekleme ve bildirim seçenekleri
\end{itemize}

\subsubsection{Detay Görünümü}

% ---- GÖRSEL 5: Detay Ekranı ----
\begin{center}
\fbox{\parbox{0.5\textwidth}{\centering\vspace{4cm}\textbf{[GÖRSEL 5: Detay Ekranı]}\\\small Borç detayları, QR kod, paylaşım butonları\vspace{4cm}}}
\end{center}

Detay ekranı işlevleri:
\begin{itemize}
    \item Tam borç bilgileri görüntüleme
    \item USD dönüşümü gösterimi
    \item ``Ödendi'' olarak işaretleme
    \item QR kod oluşturma
    \item Sosyal medya paylaşımı
    \item Silme işlemi
\end{itemize}

\subsection{Tasarım Tercihleri}

\begin{itemize}
    \item \textbf{Renk Kodlaması:} Yeşil-kırmızı renk ayrımı ile alacak/borç görsel olarak ayırt edilmektedir
    \item \textbf{Tipografi:} Roboto font ailesi, farklı boyutlarda (32sp tutar, 18sp başlık, 14sp açıklama)
    \item \textbf{Elevation:} CardView bileşenlerinde 4dp gölge efekti
    \item \textbf{Ödenen Kayıtlar:} Üstü çizili metin ve \%50 opaklık ile görsel ayrım
    \item \textbf{RTL Desteği:} Sağdan sola yazılan diller için uyumluluk
\end{itemize}

%-------------------------------------------
% 6. UYGULAMANIN ÇALIŞMA PRENSİBİ
%-------------------------------------------
\section{Uygulamanın Çalışma Prensibi}

\subsection{Uygulama Akışı}

\begin{enumerate}
    \item Kullanıcı uygulamayı açtığında \textbf{DashboardActivity} başlatılır
    \item Dashboard, veritabanından toplam alacak ve borç değerlerini çeker
    \item \textbf{CurrencyService} aracılığıyla güncel USD kuru alınır
    \item Kullanıcı karta tıkladığında \textbf{MainActivity}'e yönlendirilir
    \item Liste ekranında \textbf{DebtAdapter} ile veriler RecyclerView'da gösterilir
    \item FAB tıklandığında \textbf{AddDebtActivity} açılır
    \item Liste öğesine tıklandığında \textbf{DebtDetailActivity} açılır
\end{enumerate}

\subsection{Veritabanı İşlemleri}

\begin{lstlisting}[language=Java, caption=Borç Ekleme İşlemi]
// AddDebtActivity.java - saveDebt() metodu
Debt debt = new Debt();
debt.setPersonName(personName);
debt.setAmount(amount);
debt.setType(type); // "RECEIVABLE" veya "PAYABLE"
debt.setDate(selectedDate);
debt.setDueDate(dueDate);
debt.setNotificationEnabled(notificationEnabled);
debt.setRecurringType(recurringType);

long id = databaseHelper.addDebt(debt);

if (notificationEnabled && dueDate > 0) {
    NotificationHelper.scheduleNotification(context, debt);
}
\end{lstlisting}

\subsection{Bildirim Mekanizması}

Uygulama, vade tarihine sahip borçlar için iki aşamalı hatırlatma sistemi kullanmaktadır:

\begin{enumerate}
    \item \textbf{1 Gün Önce:} Sabah 09:00'da ön hatırlatma
    \item \textbf{Vade Günü:} Sabah 10:00'da son hatırlatma
\end{enumerate}

% ---- GÖRSEL 6: Bildirim Akış Diyagramı ----
\begin{center}
\fbox{\parbox{0.8\textwidth}{\centering\vspace{2cm}\textbf{[GÖRSEL 6: Bildirim Akış Diyagramı]}\\\small AlarmManager $\rightarrow$ AlarmReceiver $\rightarrow$ Notification\vspace{2cm}}}
\end{center}

\subsection{Döviz Kuru Entegrasyonu}

\begin{lstlisting}[language=Java, caption=Döviz Kuru Çekme]
// CurrencyService.java
public void getExchangeRate(ExchangeRateCallback callback) {
    // Onbellek kontrolu (1 saat gecerlilik)
    if (isCacheValid()) {
        callback.onSuccess(cachedRate);
        return;
    }

    // API cagrisi
    retrofit.getLatestRates().enqueue(new Callback<>() {
        @Override
        public void onResponse(...) {
            double rate = response.body().getRates().get("TRY");
            cacheRate(rate);
            callback.onSuccess(rate);
        }
    });
}
\end{lstlisting}

\subsection{Olay Tetikleyicileri}

\begin{table}[H]
\centering
\caption{Ana Olay Tetikleyicileri}
\begin{tabular}{|l|l|p{5cm}|}
\hline
\textbf{Olay} & \textbf{Metot} & \textbf{İşlev} \\
\hline
Uygulama Açılışı & onCreate() & Veritabanı bağlantısı, UI başlatma \\
\hline
Kart Tıklama & onClick() & İlgili Activity'e yönlendirme \\
\hline
FAB Tıklama & onClick() & AddDebtActivity başlatma \\
\hline
Kaydet Butonu & onClick() & Veritabanına kayıt ekleme \\
\hline
Liste Öğesi & onItemClick() & Detay sayfasına geçiş \\
\hline
Arama Metni & onTextChanged() & Liste filtreleme \\
\hline
Tab Değişimi & onTabSelected() & Liste türü değiştirme \\
\hline
\end{tabular}
\end{table}

%-------------------------------------------
% 7. PERFORMANS VE VERİMLİLİK
%-------------------------------------------
\section{Performans ve Verimlilik}

\subsection{Veritabanı Performansı}

SQLite veritabanı sorguları optimize edilmiş olup, aşağıdaki performans değerleri elde edilmiştir:

\begin{table}[H]
\centering
\caption{Veritabanı Sorgu Süreleri}
\begin{tabular}{|l|l|}
\hline
\textbf{İşlem} & \textbf{Ortalama Süre} \\
\hline
Tekil kayıt ekleme & $\sim$15 ms \\
\hline
Tüm kayıtları listeleme (100 kayıt) & $\sim$45 ms \\
\hline
Toplam hesaplama (SUM sorgusu) & $\sim$8 ms \\
\hline
Arama sorgusu (LIKE) & $\sim$25 ms \\
\hline
\end{tabular}
\end{table}

\subsection{API Önbellekleme}

Döviz kuru API çağrıları için \textbf{1 saatlik önbellek} mekanizması uygulanmıştır:

\begin{itemize}
    \item İlk çağrıda API'den veri alınır ve \texttt{SharedPreferences}'a kaydedilir
    \item Sonraki çağrılarda önbellek süresi kontrol edilir
    \item Önbellek geçerliyse API çağrısı yapılmaz
    \item Ağ hatası durumunda son geçerli değer kullanılır
\end{itemize}

\subsection{Bellek Kullanımı}

\begin{itemize}
    \item \textbf{RecyclerView ViewHolder Pattern:} Liste öğeleri için bellek optimizasyonu
    \item \textbf{Singleton Pattern:} CurrencyService ve PreferenceManager için tek örnek
    \item \textbf{Lazy Loading:} Veriler yalnızca gerektiğinde yüklenir
\end{itemize}

\subsection{Test Senaryoları}

\begin{enumerate}
    \item \textbf{Çoklu Kayıt Testi:} 500 kayıt ile liste performansı test edildi - akıcı kaydırma
    \item \textbf{Ağ Kesintisi Testi:} Çevrimdışı modda önbellek başarıyla kullanıldı
    \item \textbf{Bildirim Testi:} Zamanlanmış bildirimler doğru zamanda tetiklendi
    \item \textbf{Cihaz Yeniden Başlatma:} BootReceiver ile alarmlar başarıyla yeniden planlandı
\end{enumerate}

%-------------------------------------------
% 8. SONUÇLAR VE YORUM
%-------------------------------------------
\section{Sonuçlar ve Yorum}

\subsection{Elde Edilen Sonuçlar}

Bu proje kapsamında, Android platformu için tam işlevsel bir borç takip uygulaması geliştirilmiştir. Uygulama, belirlenen tüm hedefleri karşılamaktadır:

\begin{itemize}
    \item[$\checkmark$] Borç ve alacak kayıtlarının yönetimi
    \item[$\checkmark$] SQLite ile yerel veri depolama
    \item[$\checkmark$] Zamanında hatırlatma bildirimleri
    \item[$\checkmark$] Güncel döviz kuru entegrasyonu
    \item[$\checkmark$] PDF ve CSV dışa aktarım
    \item[$\checkmark$] QR kod ve sosyal paylaşım
    \item[$\checkmark$] Material Design 3 uyumlu arayüz
\end{itemize}

\subsection{Öğrenilen Deneyimler}

\begin{enumerate}
    \item \textbf{AlarmManager Kullanımı:} Android 12+ sürümlerinde \texttt{SCHEDULE\_EXACT\_ALARM} izninin gerekliliği öğrenildi
    \item \textbf{Retrofit Entegrasyonu:} Asenkron API çağrılarının yönetimi ve hata işleme deneyimi kazanıldı
    \item \textbf{Material Design:} CardView, TabLayout, FloatingActionButton gibi bileşenlerin etkin kullanımı
    \item \textbf{FileProvider:} Android 7+ sürümlerinde dosya paylaşımı için güvenli yöntemler
\end{enumerate}

\subsection{Geliştirme Önerileri}

\subsubsection{Kısa Vadeli İyileştirmeler}
\begin{itemize}
    \item Karanlık mod (Dark Theme) desteği
    \item Birden fazla para birimi desteği
    \item Grafik ve istatistik görünümü
    \item Widget desteği
\end{itemize}

\subsubsection{Uzun Vadeli Planlar}
\begin{itemize}
    \item Firebase entegrasyonu ile bulut senkronizasyonu
    \item Çoklu cihaz desteği
    \item Biyometrik kimlik doğrulama (parmak izi)
    \item Otomatik yedekleme ve geri yükleme
    \item Grup borç takibi özelliği
\end{itemize}

\subsection{Sonuç}

Borç Takip uygulaması, kişisel finans yönetimi alanında pratik bir çözüm sunmaktadır. Proje sürecinde Android geliştirme, veritabanı yönetimi, API entegrasyonu ve kullanıcı arayüzü tasarımı konularında değerli deneyimler edinilmiştir. Uygulama, gelecekte eklenecek özelliklerle daha kapsamlı bir finans yönetim aracına dönüştürülme potansiyeline sahiptir.

%-------------------------------------------
% KAYNAKÇA
%-------------------------------------------
\section{Kaynakça}

\begin{enumerate}
    \item Griffiths, D., \& Griffiths, D. (2021). \textit{Head First Android Development: A Learner's Guide to Building Android Apps with Kotlin} (3rd ed.). O'Reilly Media.

    \item Android Developers. (2025). \textit{Android Developer Documentation}. Google. \url{https://developer.android.com/docs}

    \item Material Design. (2025). \textit{Material Design 3 Guidelines}. Google. \url{https://m3.material.io/}

    \item Square, Inc. (2024). \textit{Retrofit: A type-safe HTTP client for Android and Java}. \url{https://square.github.io/retrofit/}

    \item Yılmaz, A., \& Demir, B. (2024). Android uygulamalarında SQLite veritabanı optimizasyonu. \textit{Türkiye Bilişim Vakfı Dergisi}, 15(3), 45-58.
\end{enumerate}

\end{document}
